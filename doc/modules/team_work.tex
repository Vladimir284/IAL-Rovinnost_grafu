% @file documentary.tex
% @project IAL Náhradní projekt - 05. Rovinnost grafu
% @author Vladimir Meciar (xmecia00)
% @brief This file id module for documentation.tex describeing work in team
% @changes 7.12.2022
\section{Práce v týmu}

\subsection{Rozdělení práce}
První s schůzku jsme měli v oktobri.
Touto dobou jsme se seznamovali a začali sme rozoberat zakladnu problematiku grafov. Sucastou bolo aj urcenie algoritmu
na odhalenie planarity grafu.
Od tejto casti sme nanestasie dlho nijako nepokrocili nakolko sme mali na starosti ine poinnosti suvisiace so skolou

Práci jsme neměli dopředu rezdělenou a úkoly jsme si rozdělovali podle potřeby v průběhu práce.

\begin{enumerate}
    \item[] Navrh a implementacia automatu pre prijmanie znakov
    \begin{itemize}[noitemsep,topsep=0pt]
        \item Ondrej Podrouzek
    \end{itemize}

    \item[] Testovanie automatu
    \begin{itemize}[noitemsep,topsep=0pt]
        \item Rastislav Mazur
    \end{itemize}

    \item[] Navrh, implementacia a testovanie datovej struktury pre ukladanie grafu
    \begin{itemize}[noitemsep,topsep=0pt]
        \item Vladimír Mečiar
    \end{itemize}


    \item[] Implementacia finalnej logiky programu
    \begin{itemize}[noitemsep,topsep=0pt]
        \item Jakub Kavka
    \end{itemize}

    \item[] Testovanie finalnej logiky programu
    \begin{itemize}[noitemsep,topsep=0pt]
        \item Rastislav Mazur
    \end{itemize}


    \item[] Tvorba dokumentace
    \begin{itemize}[noitemsep,topsep=0pt]
        \item Vladimír Mečiar
    \end{itemize}

    \item[] Revizia kodu a dokumentacie
    \begin{itemize}[noitemsep,topsep=0pt]
        \item Rastislav Mazur, Ondrej Podrouzek
    \end{itemize}

\end{enumerate}


\subsection{Verzování}
Pro sdílení a verzování kódu jsme použili verzovací systém GitHub.

V projektu jsme pracovali na více částech současně.
Pro každou větší část jsme vytvářeli vlastní větve, které jsme po dokončení mergovali do větve \texttt{develop}.
Do finální verze repozitara \texttt{main} jsme prepojili až finální otestovanou verzi a následně jsme přímo v ní prováděli jen poslední úpravy.

\subsection{Komunikace}
Jako komunikační platformy jsme využili \textit{Discord}, který sloužil jako hlavní komunikační kanál.
Pro sdílení učebních materiálů jsme vytvořili \texttt{Discord} server.